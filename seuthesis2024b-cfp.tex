\documentclass[a4paper, 12pt]{article}
\usepackage[fontset = none, zihao={-4}]{ctex}
\usepackage{anyfontsize}
\usepackage[dvipsnames]{xcolor}
\usepackage{tikz}
\usepackage[margin=20mm, top=72mm]{geometry}
\usepackage{shadowtext}
\usepackage{hologo}
\usepackage[many]{tcolorbox}
\usepackage{varwidth}
\usepackage{lua-ul}
\usepackage{hyperref}

\setmainfont{TeX Gyre Termes}
\setsansfont{TeX Gyre Heros}
\setmonofont[Scale = MatchLowercase]{Menlo}
\setCJKmainfont{Source Han Serif SC}[AutoFakeSlant]
\setCJKsansfont{Source Han Sans SC}
\setCJKmonofont{Source Han Mono SC}

\usetikzlibrary{positioning}

\definecolor{seuGreen}{HTML}{587558}
\definecolor{seuYellow}{HTML}{fdd000}

\newcommand\themeColor{seuGreen} % Brown, ForestGreen, Blue, Bittersweet
\newcommand\secondColor{seuYellow}

\shadowcolor{\themeColor!60!black}
\shadowoffset{2.5pt}

\newtcolorbox{warning}[1][]{enhanced,
  before skip=2mm,after skip=3mm,
  boxrule=0.4pt,left=5mm,right=2mm,top=1mm,bottom=1mm,
  colback=\secondColor!25,
  colframe=\secondColor!70!black,
  sharp corners,rounded corners=southeast,arc is angular,arc=3mm,
  underlay={%
    \path[fill=tcbcolback!80!black] ([yshift=3mm]interior.south east)--++(-0.4,-0.1)--++(0.1,-0.2);
    \path[draw=tcbcolframe,shorten <=-0.05mm,shorten >=-0.05mm] ([yshift=3mm]interior.south east)--++(-0.4,-0.1)--++(0.1,-0.2);
    \path[fill=\secondColor!70!black,draw=none] (interior.south west) rectangle node[white]{\Huge\bfseries !} ([xshift=4mm]interior.north west);
    },
  drop fuzzy shadow,#1}

\hypersetup{
  , colorlinks = true
  , linkcolor = \themeColor
  , urlcolor = \themeColor
  , citecolor = \themeColor
  , pdfauthor = {Wuqiong Zhao}
  , pdftitle = {seuthesis2024b Call for Papers}
}

\pagestyle{empty}

\ExplSyntaxOn
\RenewDocumentCommand \section { m m }
  {
    \phantomsection
    \vspace*{1em}
    \addcontentsline{toc}{section}{ #1 }
    {
      \Large\sffamily\color{\themeColor}
      \noindent
      {\rule[-.35em]{.4em}{1.2em}}~\textbf{\vphantom{pl}#1} \hfill {\large #2} \par
      \vspace*{-.3em}
      \noindent\rule[.38em]{\textwidth}{.8pt}
    }
    \par
  }
\ExplSyntaxOff

\begin{document}

%% Title
\begin{tikzpicture}[remember picture, overlay, transform shape]
  \node [anchor=north west, inner sep=0pt]
    at (current page.north west)
    {
      \begin{tikzpicture}[
        , title name/.style = {
          , minimum width = \paperwidth
          , minimum height = 60mm
          , shading = axis
          , left color = NavyBlue!20!black
          , right color = TealBlue!20!black
          , shading angle = 45
          , align = center
        }
        , cfp/.style = {
          , minimum width = \paperwidth
          , minimum height = 20mm
          , fill = \themeColor
          , font = \color{white}\sffamily\bfseries\Huge
        }
        , outer sep = 0
      ]
        \node (title-name) [title name] {
          \scalebox{1.8}{%
            \ttfamily\large
            \textcolor{LimeGreen}{\string\documentclass}%
            \textcolor{Orange}{\{}%
            \textcolor{Yellow}{\textbf{\underline{seuthesis2024b}}}%
            \textcolor{Orange}{\}}%
          }\\[5mm]
          \textcolor{gray!30}{\Large\textsf{东南大学 2024 届本科毕设 \textrm{\LaTeX{}} 模板}}%
          \vspace*{-5mm}
        }; 
        \node [cfp, below = 0mm of title-name] {
          \shadowtext{Call for Papers\quad 论文征集}
        };
      \end{tikzpicture}
    };
\end{tikzpicture}

%% Bottom
\begin{tikzpicture}[remember picture, overlay, transform shape]
  \node [
    , anchor = south west
    , inner sep = 0pt
    , minimum width = \paperwidth
    , minimum height = 10mm
    , fill = \themeColor
    , font = \color{gray!30}\small
  ] at (current page.south west) {\copyright{} 2024 \href{https://wqzhao.org}{\color{gray!30}Wuqiong Zhao}};
\end{tikzpicture}

\section{Template Usage}{模板使用}

\texttt{seuthesis2024b} 模板使用最新 \LaTeX3 语法编写,需使用 \hologo{XeLaTeX} 引擎编译。
模板已在 GitHub 开源:
\href{https://github.com/Teddy-van-Jerry/seuthesis2024b}{\ttfamily github.com/Teddy-van-Jerry/seuthesis2024b}。
此 \LaTeX{} 模板依照官方 MS Word 模板制作,功能支持包括:
\textbf{图片}(含子图)、\textbf{表格}(含跨页表格),\textbf{参考文献}、\textbf{数学公式}、\textbf{脚注}、自动\textbf{PDF元数据}等,
并提供\textbf{多种文档选项}供用户选择。
使用此模板撰写毕业论文,可使论文格式更加规范,提高撰写效率!

\begin{warning}
  此\textbf{非官方(unofficial)}模版基于《东南大学本科毕业设计(论文)参考模板 (2024年1月修订)》设计。
  毕设论文具体格式要求请使用者\underLine{自行确认}。
\end{warning}

\section{Topics of Interest}{征集主题}
\textbf{你的毕设选题}。包括但不局限于:
\href{https://wqzhao.org/pub/zhao2023ompl}{智能反射表面(RIS)信道估计}、
\href{https://wqzhao.org/pub/you2023beam}{波束与反射模式联合设计}、
\href{https://wqzhao.org/pub/zhao2023flexible}{高层级综合(HLS)}、
\href{https://wqzhao.org/pub/zhao2023automatic}{硬件自动生成语言设计}。
% \textcolor{\secondColor}{(此处夹带私货)}

\section{Submission Procedure}{投稿流程}

请使用东南大学毕业设计(论文)智能管理平台 \href{https://bysj.seu.edu.cn}{\ttfamily bysj.seu.edu.cn} 提交论文。
此 \LaTeX{} 模板使用流程可参考 \href{https://github.com/Teddy-van-Jerry/seuthesis2024b}{GitHub 仓库}相关说明。
推荐使用 \texttt{latexmk} 工具编译或 Overleaf 平台。

\section{Important Dates}{重要日期}

请参考东南大学教务处和各学院通知。

\section{Contact}{联系方式}

\noindent
\textbf{General Chair}: Wuqiong Zhao, \textit{Southeast University, China} (e-mail: \href{mailto:me@wqzhao.org}{\ttfamily wqzhao.org@seu.edu.cn})

\end{document}
